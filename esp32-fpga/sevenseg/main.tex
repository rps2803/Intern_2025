\subsection{Seven-Segment Display}
This subsection illustrates the use of SPI Communication between the Vaman-ESP32
and the FPGA onboard the Vaman-Pygmy by controlling a seven segment display 
remotely.

\subsubsection{Components}
The components required are listed in
\autoref{tab:esp32-fpga-sevenseg-components}.
\begin{table}[!ht]
    \centering
    \input{inter-chip/esp32-fpga/sevenseg/tables/components.tex}
    \caption{Components Required for Controlling the Seven-Segment Display via SPI.}
    \label{tab:esp32-fpga-sevenseg-components}
\end{table}

\subsubsection{Connections}
The connections to be made on the Vaman board are listed in
\autoref{tab:esp32-fpga-led-connections}. The connections to be made with the
seven segment display are listed in
\autoref{tab:esp32-fpga-sevenseg-display-connections}.

\begin{table}[!ht]
    \centering
    \input{inter-chip/esp32-fpga/sevenseg/tables/sevenseg-connections.tex}
    \caption{Connections to Interface a Seven-Segment Display with Vaman-Pygmy.}
    \label{tab:esp32-fpga-sevenseg-display-connections}
\end{table}

\subsubsection{Building}
\begin{enumerate}
    \item Build the PlatformIO project at
    \begin{lstlisting}
inter-chip/esp32-fpga/sevenseg/codes/esp32
    \end{lstlisting}
    \item Flash the project .bin file using USB-UART connected to the 
    Vaman-ESP32, via PlatformIO or ArduinoDroid.
    \item Build the FPGA project .bin file by entering the following commands at
    a terminal window.
    \begin{lstlisting}
# The following variable can be sourced from .shellrc or .venv/bin/activate
export QORC_SDK_PATH=/path/to/pygmy-sdk
cd inter-chip/esp32-fpga/sevenseged/codes/fpga
make
    \end{lstlisting}
    \item On building successfully, the .bin file is generated at
    \begin{lstlisting}
inter-chip/esp32-fpga/sevenseg/codes/fpga/rtl/AL4S3B_FPGA_Top.bin
    \end{lstlisting}
    \item Flash the .bin file to the Vaman-Pygmy by resetting it to bootloader
    mode and entering the following command at a terminal window,
    \begin{lstlisting}
python3 /path/to/tinyfpga-programmer-gui.py --port /dev/ttyACMxx --mode fpga --appfpga /path/to/AL4S3B_FPGA_Top.bin --reset
    \end{lstlisting}
    where /dev/ttyACMxx is the port at which the Vaman board is available. This
    can be obtained by inspecting the output of the following command (requires
    root/sudo privileges).
    \begin{lstlisting}
dmesg -w
    \end{lstlisting}
\end{enumerate}

\subsubsection{Demonstration}
\begin{enumerate}[resume]
    \item Find the IP address of the Vaman-ESP32 by inspecting the output of the
    serial terminal, or by typing at a terminal window
    \begin{lstlisting}
ifconfig
nmap -sn xx.yy.zz.0/24
    \end{lstlisting}
    where xx.yy.zz represents the first three octets of the IP address of your
    device on the WiFi network interface, found using \texttt{ifconfig}.
    \item Then, go to the following website and interact with the HTML form to
    see the output on the seven-segment display.
    \begin{lstlisting}
http://<VAMAN-IP>/sevenseg
    \end{lstlisting}
\end{enumerate}

\subsubsection{Exercises}
\begin{enumerate}[resume]
    \item Modify the ESP32 code to have a radio button for each segment.
    Optionally, add another radio button for the dot.
    \item Try to minimize the number of SPI transactions and the amount of data
    transmitted in each of them.
\end{enumerate}